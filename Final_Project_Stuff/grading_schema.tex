\documentclass[a4paper,12pt]{report}

\usepackage[margin=2cm,top=2.5cm,headheight=16pt,headsep=0.1in,heightrounded]{geometry}
\usepackage{array}
\usepackage[utf8]{inputenc}
\usepackage{multirow}
\usepackage{longtable}
\usepackage{fancyhdr}
    \pagestyle{fancy}

\lhead{Scientific Programming in Python}
\rhead{SS 2022}

\usepackage{sectsty}

\sectionfont{\centering}
\begin{document}
\section*{Final Project Grading Schema}
\begin{itemize}
  \item Project registration until 11.06., hand in until 09.07., grading until 31.08.
  \item If you are not satisfied with the grade, you can always choose to take a “pass” instead
\end{itemize}
Projects should be handed in as GitHub repositories with README.md files and Python code in
.py files. It’s also possible to use Jupyter Notebooks, but only to demonstrate e.g. usage or
interpretation of the outputs, not for the main body of the code. You can assume that the user
has a local system with working Python 3.7, Miniconda, and Git installations and basic
knowledge of how to use them.
The following grading schema is going to be the guideline for the grading of the final project.
The finer distinctions within design, implementation, and documentation may not be applicable
to every project. If that’s the case, we will communicate this when the project is registered.

%Wow, latex tables are a real pain in the ass....
\begin{center}
\begin{tabular}{ | m{0.3\textwidth} | m{0.5\textwidth}| m{0.2\textwidth} | } 
  \hline
  \multirow{2}{4em}{\textbf{Design}} & \vspace{4mm} Suitable inputs, outputs, methods were chosen.
  \begin{itemize}
    \item Outputs relevant for project goal
    \item Inputs relevant for project goal
    \item Methods adequate for project goal
  \end{itemize}
  & 6 points \\ 
  \cline{2-3}
  & \vspace{4mm} Project is - to some degree - customizable.
  \begin{itemize}
    \item User can select different settings or inputs
  \end{itemize}
  & 4 points \\ 
  \hline
  \multirow{4}{4em}{\textbf{Implementation}} & \vspace{4mm} The project is structured sensibly overall into folders and files. 
  \begin{itemize}
    \item Separate folders for inputs and outputs
    \item README.md at top level
    \item Separate file for requirements and settings where appropriate
  \end{itemize}
  & 3 points \\ 
  \cline{2-3}
  & \vspace{4mm} The code is structured sensibly.
  \begin{itemize}
    \item Split up different functionality into different files and functions
    \item As little code redundancy as possible
    \item Bonus points for elegant solutions
  \end{itemize}
  & 5 points \\
  \cline{2-3}
  & \vspace{4mm} The code is executable and yields the expected outputs. \vspace{3mm}
  & 6 points \\
  \hline  
  \end{tabular}
  \begin{tabular}{ | m{0.3\textwidth} | m{0.5\textwidth}| m{0.2\textwidth} | } 
  \hline
  & \vspace{4mm} The code is the most efficient solution available.
  \begin{itemize}
    \item Known library functions for computationally intensive operations are used
    \item There are no obvious inefficiencies (endless loops, functionality duplication)
  \end{itemize}
  & 6 points \\ 
  \hline
  \multirow{4}{4em}{\textbf{Documentation}} & \vspace{4mm} There is a comprehensive overview of the project's goal, motivation and structure \vspace{3mm}
  & 5 points \\ 
  \cline{2-3}
  & \vspace{4mm} There are instructions for setting up the project on a local machine.
  \begin{itemize}
    \item Requirements are clearly described
    \item Download links for external resources
  \end{itemize}
  & 3 points \\
  \cline{2-3}
  & There are instructions for the intended usage of the project.
  \begin{itemize}
    \item Commands to run the project
    \item Information on selecting inputs
    \item Information on selecting settings
    \item Examples for interpreting possible outputs
  \end{itemize}
  & 6 points \\
  \cline{2-3}
  & The code is well-documented.
  \begin{itemize}
    \item Docstrings for all functions
    \item Comments for complex pieces of code
    \item Sensible variable names
  \end{itemize}
  & 6 points \\ 
  \hline
\end{tabular}
\end{center}
There are 50 points in total: 10 for design, 20 for implementation, and 20 for documentation.
The total sum of points will be used to determine the numerical grade that each person in the
group will receive for the whole course. Of course, you can also do a final project on your own
without a group.

\begin{center}
\begin{tabular}{ |c|c|c|c|c| } 
 \hline
 $>=$ 47.5 & 1.0 & & $>=$ 35.0 & 2.7 \\ \hline
 $>=$ 45.0 & 1.3 & & $>=$ 32.5 & 3.0  \\ \hline
 $>=$ 42.5 & 1.7 & & $>=$ 30.0 & 3.3   \\ \hline
 $>=$ 40.0 & 2.0 & & $>=$ 27.5 & 3.7  \\ \hline
 $>=$ 37.5 & 2.3 & & $>=$ 25.0 & 4.0  \\ \hline
\end{tabular}
\end{center}

\end{document}